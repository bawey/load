\subsection{RCMS framework} \label{subsec:framework}
A somewhat old version of the framework is required, in the development environment it was the revision 8055 of \texttt{https://svn.cern.ch/reps/rcms/rcms/trunk/framework}.


\subsection{Hardware Configuration Database}
\label{subsec:hwconfdb}
Hardware Configuration DB connection is required for the \texttt{HwInfo} class to work. \texttt{HwInfo} provides helper methods that can be used in EPL statements to retrieve information about the hardware.
In development environment the connection was setup using a port-forwarding option in \texttt{~/.ssh/config}: \texttt{LocalForward 10121 cmsrac11-v:10121}, a hard-coded DB url: \texttt{jdbc:oracle:thin:@localhost:10121/cms\_omds\_tunnel.cern.ch} and an active ssh connection with \emph{cmsusr}. Building the project within \emph{Eclipse} requires the \emph{framework} project as dependency (trunk version as of 26.06.2014).

\subsection{proxy}
\label{subsec:proxy}
Fetching flashlists as they are published was achieved via configuring ssh (\texttt{DynamicForward 1080} option), specifying the address to fetch the flashlists from and settings for SOCKS Proxy:
\begin{lstlisting}
socksProxyHost=127.0.0.1
proxySet=true
socksProxyPort=1080
\end{lstlisting}

\subsection{Events Database}
\label{subsec:eventsdb}
The Events Database was setup during development to facilitate events playback, especially any partial playback involving arbitrary start and end timestamp.
Database structure automatically mimicked the structure of published flashlists, ie. a table was created for each flashlist type and a column for each flashlist field. 
For convenience, all fields were stored as \texttt{VARCHAR} and one additional numeric column was added to each table: \texttt{fetchstamp} indicating the timestamp of fetching the flashlist. An extra table \texttt{fetchstamps} stores all the unique \texttt{fetchstamp} values for all other tables. Also, tables storing useful events (ie. used for analyses) have been indexed by \texttt{fetchstamp} column. This minimizes "query misses" - attempts to retrieve data for a \texttt{fetchstamp} with no corresponding events.

Password can be specified using the \texttt{flashlistDbPass} parameter. \emph{MySQL} is the only fully supported database type with some rudimentary \emph{MongoDB} also exists, but needs to be worked on to be useful.






