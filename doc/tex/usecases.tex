\subsection{Offline analysis}
In order to perform offline analysis, program has to be configured to use the \emph{Events Database}[\ref{subsec:eventsdb}] and a \emph{Hardware Configuration Database} connection [\ref{subsec:hwconfdb}] needs to be set up.

\begin{lstlisting}
flashlistDbMode=read
flashlistDbType=mysql
flashlistDbHost=localhost
flashlistDbUser=load
flashlistDbName=flashlists_rest
\end{lstlisting}


\subsection{Online analysis}
Online analysis requires a slightly different approach: \emph{Events Database} should be disabled in favor of of providing \texttt{onlineFlashlistsRoot} entries pointing to a network location to retrieve the flashlists from. 
Connection to the \emph{Hardware Configuration Database}[\ref{subsec:hwconfdb}] remains a requirement, while a SOCKS proxy[\ref{subsec:proxy}] configuration is also needed.

\begin{lstlisting}[caption={Sample options for fetching flashlists over the network}]
onlineFlashlistsRoot[0]=http://srv-c2d04-19.cms:9941/urn:xdaq-application:lid=400/
onlineFlashlistsRoot[1]=http://srv-c2d04-19.cms:9942/urn:xdaq-application:lid=400/
\end{lstlisting}


\subsection{Dumping data for offline analyses}
\subsubsection{Dumping online data}
Originaly the data was saved to and played back from files only. Thus, no mechanism has been implemented to dump the data directly into a database. Instead, provided that flashlists location is supplied and reachable, their contents are dumped into the folder specified by configuration option \texttt{THAT\_IS\_CURRENTLY\_LOST\_ALONG\_WITH\_THE\_CODE\_THAT\_USES\_IT}. Within that directory a subdirectory is creatred for each flashlist type. Flashlist rows are dumped into the files named 0, 1, 2... and so on - unique rows only, switching to the next file once the current one reaches the size of 256MB. Dumping also involves adding information about the time of fetching the flashlist - the \texttt{fetchstamp} column.

\begin{lstlisting}[caption={Sample directory structure of dumped flashlists data}]
.
+-- urn:xdaq-flashlist:BU
|   +-- 0
|   +-- 1
|   +-- 2
|   +-- 3
|   +-- 4
|   +-- 5
|   └-- 6
+-- urn:xdaq-flashlist:diskInfo
|   +-- 0
|   +-- 1
|   +-- 10
|   +-- 2
|   +-- 3
|   +-- 4
|   +-- 5
|   +-- 6
|   +-- 7
|   +-- 8
|   └-- 9
(...)
+-- urn:xdaq-flashlist:StorageManagerPerformance
    └-- 0

\end{lstlisting}

\subsubsection{Populating database}
Switching \texttt{flashlistDbMode} parameter to \texttt{write} makes the \texttt{Load} class \texttt{main} method invoke the \texttt{main} method of the dumper to pass control into what used to be a separate application. 
Apart from 

\begin{lstlisting}
flashlistDbMode=write
flashlistDbType=mysql
flashlistDbHost=myDbHost
flashlistDbUser=myDbUser
flashlistDbName=myDbName
flashlistDbPass=myDbPass


flashlistForDbDir[0]=/depot/flashlists13.11/
flashlistForDbDir[1]=/depot/flashlists13.11_2/
flashlistForDbDir[2]=/depot/flashlists13.11_3/
flashlistForDbDir[0]=/depot/flashlists13.11_4/
\end{lstlisting}

The configuration above assumes each of \emph{root} flashlist directories contains a set of subdirectories named after the flashlist they hold the values from. Listing below shows an example of such directory structure.





	\begin{itemize}
		\item Event DB connection for writing
		\item Flashlists dumped on the disk
	\end{itemize}
\subsection{Dumping online data}
As mentioned above, the flashlists were firs dumped into files and only afterwards into a databases.
	\begin{itemize}
		\item Online flashlists connection
	\end{itemize}


