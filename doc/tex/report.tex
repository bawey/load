\documentclass[11pt,oneside,a4paper]{article}
\usepackage[margin=1in,footskip=0.25in]{geometry}
\usepackage{listings}
\usepackage{ltablex}
\usepackage{multirow}

\newcolumntype{Y}{>{\centering\arraybackslash}X}
\renewcommand{\arraystretch}{2}

\lstset{
	basicstyle=\ttfamily\small,
    keywordstyle=\color{blue}\ttfamily,
    stringstyle=\color{red}\ttfamily,
    commentstyle=\color{green}\ttfamily,
    breaklines=true,
	showstringspaces=false,
	tabsize=2
}


\title{Prototype of Esper-based DAQ expert system}
\date{\today}
\author{Tomasz Bawej (tomek.bawej@gmail.com)}


\begin{document}

\maketitle

\tableofcontents
\clearpage

\section{Overview}
\subsection {Goal}
\textbf{L}evel \textbf{0} \textbf{A}nomaly \textbf{D}etective, commonly abbreviated as L0AD, has been implemented in an effort to evaluate Esper as a potential backbone of the CMS DAQ expert system. The project consists of a collection of EPL scripts for processing events as well as components responsible for definition and retrieval of events, registration of EPL statements and displaying the outcome of analyses.
The project has never reached a mature state, hence its structure is far from perfect.
\subsection{Data sources}
The program receives input data in the form of events. While Esper allows several possible event representations, logically a single event is a collection of labelled data, a map. For an event to be fed into the Esper engine it has to be delivered in a predefined stream. Streams define the structure of events they can carry so that all events entering a particular stream would contain fields of the same names and types. \\

In L0AD, the definition and continuous population of event streams is handled by \texttt{EventsTap} objects. The project contains two implementations of taps: one for retrieving the events from a database and another for fetching them as they are published on the web. More implementations could be added by subclassing the \texttt{EventsTap} or any of its descendants.

\subsection{Data input and processing}

Once the event streams are defined, the engine can be instructed to detect occurrences of specific circumstances. Statement objects represent instructions for the engine and Esper provides a bidirectional conversion mechanism between a statement object and its textual EPL representation. Thus, all the queries have been saved in epl files under the \texttt{epl} directory. At runtime these files are deployed as modules by a \texttt{FileBasedEplProvider} object that relies on Esper's capabilities in that matter. \texttt{EventProcessor} class plays a role of Epser event processing engine's wrapper that facilitates certain manipulations and handles certain initialization tasks, e.g. registration of helper methods available from the EPL level.

\subsection{Output}

The project introduces an \texttt{EventsSink} abstract class to handle outputting the information from the Esper event processing engine. Provided implementations include \texttt{FileSink}, \texttt{SwingGui} and \texttt{ThroughputMonitor} classes, where \texttt{ThroughputMonitor} only outputs the performance-related information. It is up to \texttt{EventsSink} subclasses to implement interface methods returning
instances of \texttt{com.espertech.esper.client.StatementAwareUpdateListener} for statements annotated with \texttt{@Verbose} and \texttt{@Watched}. 
\texttt{EventProcessor}'s constructor makes sure that every registered statement annotated with \texttt{@Verbose} or \texttt{@Watched} gets connected to compatible \texttt{EventSink}s, i.e. the ones providing appropriate listeners. This way \texttt{EventSink}s car receive statement-related updates and produce the output.


\subsection{Helper and convenience classes}

Apart from the core classes listed above, the project comprises numerous utility classes with the most notable being:

\begin{itemize}
	\item \texttt{FieldTypeResolver} - a class storing (hardcoded) information about the data types of particular flashlists' fields. It also facilitates the conversions.
	\item \texttt{Trx} - a set of convenience methods callable from EPL statements.
	\item \texttt{HwInfo} - as above, but related to CMS DAQ hardware.
	\item \texttt{Settings} - a global singleton for handling the configuration values.
	\item \texttt{LoadLogCollector} - a class collecting Esper logs and dispatching them to registered \texttt{LogSink}s. It is used to display the results and the logs side-by-side in the same GUI.
\end{itemize}

\section{Use cases}\label{sec:usage}
\subsection{Offline analysis}
In order to perform offline analysis, program has to be configured to use the \emph{Events Database}[\ref{subsec:eventsdb}] and a \emph{Hardware Configuration Database} connection [\ref{subsec:hwconfdb}] needs to be set up.

\begin{lstlisting}
flashlistDbMode=read
flashlistDbType=mysql
flashlistDbHost=localhost
flashlistDbUser=load
flashlistDbName=flashlists_rest
\end{lstlisting}


\subsection{Online analysis}
Online analysis requires a slightly different approach: \emph{Events Database} should be disabled in favor of of providing \texttt{onlineFlashlistsRoot} entries pointing to a network location to retrieve the flashlists from. 
Connection to the \emph{Hardware Configuration Database}[\ref{subsec:hwconfdb}] remains a requirement, while a SOCKS proxy[\ref{subsec:proxy}] configuration is also needed.

\begin{lstlisting}[caption={Sample options for fetching flashlists over the network}]
onlineFlashlistsRoot[0]=http://srv-c2d04-19.cms:9941/urn:xdaq-application:lid=400/
onlineFlashlistsRoot[1]=http://srv-c2d04-19.cms:9942/urn:xdaq-application:lid=400/
\end{lstlisting}


\subsection{Dumping data for offline analyses}
\subsubsection{Dumping online data}
Originaly the data was saved to and played back from files only. Thus, no mechanism has been implemented to dump the data directly into a database. Instead, provided that flashlists location is supplied and reachable, their contents are dumped into the folder specified by configuration option \texttt{THAT\_IS\_CURRENTLY\_LOST\_ALONG\_WITH\_THE\_CODE\_THAT\_USES\_IT}. Within that directory a subdirectory is creatred for each flashlist type. Flashlist rows are dumped into the files named 0, 1, 2... and so on - unique rows only, switching to the next file once the current one reaches the size of 256MB. Dumping also involves adding information about the time of fetching the flashlist - the \texttt{fetchstamp} column.

\begin{lstlisting}[caption={Sample directory structure of dumped flashlists data}]
.
+-- urn:xdaq-flashlist:BU
|   +-- 0
|   +-- 1
|   +-- 2
|   +-- 3
|   +-- 4
|   +-- 5
|   └-- 6
+-- urn:xdaq-flashlist:diskInfo
|   +-- 0
|   +-- 1
|   +-- 10
|   +-- 2
|   +-- 3
|   +-- 4
|   +-- 5
|   +-- 6
|   +-- 7
|   +-- 8
|   └-- 9
(...)
+-- urn:xdaq-flashlist:StorageManagerPerformance
    └-- 0

\end{lstlisting}

\subsubsection{Populating database}
Switching \texttt{flashlistDbMode} parameter to \texttt{write} makes the \texttt{Load} class \texttt{main} method invoke the \texttt{main} method of the dumper to pass control into what used to be a separate application. 
Apart from 

\begin{lstlisting}
flashlistDbMode=write
flashlistDbType=mysql
flashlistDbHost=myDbHost
flashlistDbUser=myDbUser
flashlistDbName=myDbName
flashlistDbPass=myDbPass


flashlistForDbDir[0]=/depot/flashlists13.11/
flashlistForDbDir[1]=/depot/flashlists13.11_2/
flashlistForDbDir[2]=/depot/flashlists13.11_3/
flashlistForDbDir[0]=/depot/flashlists13.11_4/
\end{lstlisting}

The configuration above assumes each of \emph{root} flashlist directories contains a set of subdirectories named after the flashlist they hold the values from. Listing below shows an example of such directory structure.





	\begin{itemize}
		\item Event DB connection for writing
		\item Flashlists dumped on the disk
	\end{itemize}
\subsection{Dumping online data}
As mentioned above, the flashlists were firs dumped into files and only afterwards into a databases.
	\begin{itemize}
		\item Online flashlists connection
	\end{itemize}




\section{Configuration}\label{sec:config}
This section introduces some overlap with section \ref{sec:usage} in an attempt to list all the configuration options and elaborate on their meaning where they have not been discussed so far and a discussion is required.

\subsubsection{Proxy parameters}

\texttt{socksProxyHost=127.0.0.1} \\
\texttt{proxySet=true} \\
\texttt{socksProxyPort=1080} \\

\subsubsection{Flashlist database settings}

\texttt{flashlistDbMode=read} \\

Actually only mySQL is fully supported as database type and only in this case the database engine setting makes sense. \\
\texttt{flashlistDbType=mysql} \\
\texttt{flashlistDbEngine=myisam} \\

\texttt{flashlistDbHost=localhost} \\
\texttt{flashlistDbUser=load} \\
\texttt{flashlistDbPass=your\_password} \\
\texttt{flashlistDbName=flashlists\_rest} \\

This option indicates which column, in each table, is the time of fetching particular row. This value is later used for playback. \\
\texttt{retrievalTimestampName=fetchstamp} \\

This option determines whether the retrieval timestamp column should be indexed when initializing a database based on the dump files \\
\texttt{flashlistDbIndexTimestamps=true} \\

\subsubsection{Online flashlists setting}

\texttt{onlineFlashlistsRoot[0]=http://srv-c2d04-19.cms:9941/urn:xdaq-application:lid=400/} \\
\texttt{onlineFlashlistsRoot[1]=http://srv-c2d04-19.cms:9942/urn:xdaq-application:lid=400/} \\

\subsubsection{Dumped flashlists}\label{subsec:dumps}
The directories listed using the option below are scanned in search of the flashlist dump files that will be used to create and fill database.

\texttt{flashlistForDbDir[0]=/depot/flashlists13.11/} \\
\texttt{flashlistForDbDir[1]=/depot/flashlists13.11\_2/} \\
\texttt{flashlistForDbDir[2]=/depot/flashlists13.11\_3/} \\
\texttt{flashlistForDbDir[0]=/depot/flashlists13.11\_4/} \\
\texttt{flashlistForDbDir[0]=/home/bawey/Desktop/flmini/flashlistsExport} \\
\texttt{flashlistForDbDir[2]=/home/bawey/Desktop/flashlists/41} \\


\subsubsection{EPL directories}
\texttt{eplDir[0]=epl} \\
\texttt{eplDir[1]=epl\_test}\\


\subsubsection{Output modules}
\texttt{view[0]=ThroughputMonitor} \\
\texttt{view[1]=FileSink} \\
\texttt{view[2]=SwingGui} \\

\subsection{Flashlists}
A list of flashlists to use. \\
\texttt{flashlists=levelZeroFM\_subsys;EVM;frlcontrollerLink;frlcontrollerCard;levelZeroFM\_static;gt\_cell\_lumiseg;FMMInput;jobcontrol;EventProcessorStatus;StorageManagerPerformance;FMMStatus;hostInfo}

Blacklisting specified fields per flashlist.\\
\texttt{blacklist\_jobcontrol=jobTable}
blacklist\_frlcontrollerCard=myrinetProcFile,myrinetBadEventNumber
blacklist\_levelZeroFM\_subsys=FEDS


\subsubsection{Others}
Defines a period to play the data back for. \\
\texttt{timerStart=1383763860000} \\
\texttt{timerEnd=1383763860800} \\

Data format is used when converting timestamp strings into numeric values. \\
\texttt{dateFormat=yyyy-MM-dd'T'HH:mm:ss.SSS} \\

Output directory is the location that \texttt{FileSink} writes into. \\
\texttt{outputDir=/home/bawey/load/} \\

Classes to be registered as log sinks
\texttt{logSinks="ch.cern.cms.load.sinks.SwingGui";}



\section{Configuration requirements}
\subsection{RCMS framework} \label{subsec:framework}
The project was confirmed to compile and run with revision 8055 of the \emph{RCMS framework} (\texttt{https://svn.cern.ch/reps/rcms/rcms/trunk/framework}). 
Some later revisions make it impossible to compile the project against the framework.

\subsection{Hardware Configuration Database}
\label{subsec:hwconfdb}
\emph{Hardware Configuration DB connection} is required for the \texttt{HwInfo} class to work. \texttt{HwInfo} provides helper methods that can be used in EPL statements to retrieve information about the hardware.
In development environment the connection was set up using a port-forwarding option in \texttt{~/.ssh/config}: \texttt{LocalForward 10121 cmsrac11-v:10121}, a hard-coded DB url: \texttt{jdbc:oracle:thin:@localhost:10121/cms\_omds\_tunnel.cern.ch} and an active ssh connection with \emph{cmsusr}.

\subsection{SOCKS proxy}
\label{subsec:proxy}
Fetching flashlists as they are published was achieved via configuring ssh (\texttt{DynamicForward 1080} option), specifying the address to fetch the flashlists from and settings for SOCKS Proxy:
\begin{lstlisting}
socksProxyHost=127.0.0.1
proxySet=true
socksProxyPort=1080
\end{lstlisting}
An active ssh connection with \emph{cmsusr} was needed, too.

\subsection{Events Database}
\label{subsec:eventsdb}
The Events Database was set up during development to facilitate events playback, especially any partial playback involving arbitrary start and end timestamp.
Database structure automatically mimicked the structure of published flashlists, i.e. a table was created for each flashlist type and a column for each flashlist field. 
For convenience, all fields were stored as \texttt{VARCHAR} and one additional numeric column was added to each table: \texttt{fetchstamp} indicating the timestamp of fetching the flashlist. An extra table \texttt{fetchstamps} stores all the unique \texttt{fetchstamp} values for all other tables. This minimizes "query misses" - attempts to retrieve data for a \texttt{fetchstamp} with no corresponding events. \\
Also, tables storing useful events (i.e. used for analyses) have been indexed by \texttt{fetchstamp} column. Otherwise the DB performance made the project unusable. 

Password can be specified using the \texttt{flashlistDbPass} parameter. \emph{MySQL} is the only fully supported database type with some rudimentary \emph{MongoDB} also exists, but needs to be worked on to be useful.









\subsection{Implemented checks}
A summary of implemented checks along with some brief notes can be found in the xml files under the \texttt{epl/xml} directory.

%\section{Overview of implemented checks}


\begin{tabularx}{1\textwidth}{|*{2}{Y|}}
\hline
Purpose of the check    	& Solution notes / related EPL files \\
\hline
\multicolumn{2}{|c|}{During an ongoing run.} \\
%\cline{1-3}
\hline
	
	\multirow{2}{0.5000\textwidth}{Message on Run Start giving: Run NR, SID, Detectors in, Feds in per detector} & 
	Subsystems in uses a custom aggretaion method that turns out to be running all the time (and not on demand) \\
	\cline{2-2}
	& \small{\texttt{runStartStop.epl}} \\

	\hline
	\multirow{2}{0.5000\textwidth}{Message on Run Stop giving: Run Nr, SID, Detectors in, (*)Feds in per detector, avg L1 rate, avg stream A rate, avg dead time (from trigger LAS), duration} &
	Same as for run start, but also: get avg L1 rate, avg stream A rate, avg dead time from trigger LAS, the duration \\
	\cline{2-2}
 	& \small{\texttt{runStartStop.epl}} \\

	\hline
	\multirow{2}{0.5000\textwidth}{Message on L1 trigger rate jump of 10\% or more} &
	EPL file creates some windows and streams useful for performing other loigic. A javascript method needs to be raplaced with a case statement \\
	\cline{2-2}
	& \small{\texttt{level1TriggerRate.epl}} \\

	\hline
	\multirow{2}{0.5000\textwidth}{Message on subsys going to any of: ERROR, RUNNING\_DEGRADED, RUNNING\_SOFT\_ERROR\_DETECTED, PAUSING, PAUSED, RESUMING} & \\
	\cline{2-2}
	& \small{\texttt{subsystemsStateChanges.epl}} \\

	\hline
	\multirow{2}{0.5000\textwidth}{Message on any state change of DAQ} & \\
	\cline{2-2}
	& subsystemsStateChanges.epl \\
	
	\hline
	\multirow{2}{0.5000\textwidth}{Message on jobcontrol flashlist not being updated after 1 minute} & \\
	\cline{2-2}
	& jobcontrolNotUpdated.epl \\
	
	\hline
	\multirow{2}{0.5000\textwidth}{FED dead-time $>$ 1\%} & Investigates backpressure on the same FED or its main FED. \\
	\cline{2-2}
	& deadtimeAndBackpressure.epl \\

	\hline
	\multirow{2}{0.5000\textwidth}{Backpressure $>$ 1\%} & \\
	\cline{2-2}
	& deadtimeAndBackpressure.epl \\

	\hline
	\multirow{2}{0.5000\textwidth}{Message on stream A $>$ 500 Hz} &
	Simplified, does not satisfy: after 10 seconds, repeat message every 10 seconds. Need to sum the last-per-context values \\
	\cline{2-2}
	& streamARate.epl \\

	\hline
	\multirow{2}{0.5000\textwidth}{FEDs fraction other} &
	A FedFractions stream is constantly filled with derivatives computed for each subsequently arriving pair of FmmInput events with the same fedId. 
	Fractions (busy+warning	+error+ready+oos) must add up to one, otherwise FED spends some time in an illegal (other) state. This fact should be reported and using integrals is preffered for calculations to avoid floating point numbers comparison. Why does it report negative fractions at the beginning? (As well as negative dTime). Never tested on positives. \\
	\cline{2-2}
	& fedFractions.epl \\
\hline
	\multicolumn{2}{|c|}{Rate 0 for $>$ 10 seconds} \\
	
	
	\hline
	\multirow{2}{0.5000\textwidth}{Check Bx alignment and print message if not aligned repeat after 10 seconds} & Fails to repeat the message every 10 seconds if the problem persists\\
	\cline{2-2}
	&  rateZeroTests.epl\\

	\hline
	\multirow{2}{0.5000\textwidth}{Check triggers(events) alignment + print message if not aligned repeat after 10 seconds} & 
		Fails to repeat the message every 10 seconds if the problem persists \\
	\cline{2-2}
	& rateZeroTests.epl \\

	\hline	
	\multirow{2}{0.5000\textwidth}{FEDs stuck in ERROR/OOS/WARNING/BUSY(anything not READY)} & 
	Written, running, never seen working yet. Using the FedFractions stream to find events of interest and SuspendedStatements window to suspend/resume the statement upon reception of RateStuckAtZeroEvent or RateFineEvent defined in the same source file as a way to broadcast the message that the expert enters a state where the run is ongoing but the rate has been 0 for long enough or that the rate is fine and experiment is running. 
ERROR/OOS/WARNING/BUSY have their corresponding fractions on FMMInput. Stuck means fraction = 1. A FED might also be stuck in an other state. Use the integrals from two consecutive events to determine that.
\\
	\cline{2-2}
	& rateZeroTests.epl\\


	\hline
	\multirow{2}{0.5000\textwidth}{List FEDs with backpressure or deadtime} & \\
	\cline{2-2}
	& rateZeroTests.epl\\


	\hline
	\multirow{2}{0.5000\textwidth}{Check if number of resyncs and the last resync event number is the same in all FEDs?} & 
	Check if the number of resyncs is the same and check if the last event (the resync was seen for) is the same. myrinetResync - number of resync events, myrinetLastResyncEvt - last event that the resync was seen for.\\
	\cline{2-2}
	& rateZeroTests.epl\\


\end{tabularx}



A 'mock' version (with additional comments) of the table below bac be found under the \texttt{epl/xml} directory.









\subsection{Inconsistencies remaining}

Given the lack of maturity of the project, this could well be the longest part of this report. The following list includes only the issues that caught my attention while writing this report.

\subsubsection{Resolving EPL dependencies by hand}
Until the proper, Esper-powered approach was taken, resolving EPL dependencies used to be handled by the means of inserting extra \texttt{\#include} statements and determining the deployment order by hand.
Some remnants of that are present in the EPL loading code or in the EPL itself and should have been removed.

\subsubsection{Configuration constants}
Some of these string literals are centrally defined in the \texttt{Settings} class, while some other are added only by the classes using them. 

\end{document}
