\documentclass[11pt,oneside,a4paper]{article}
\usepackage[margin=1in,footskip=0.25in]{geometry}
\usepackage{listings}
\usepackage{ltablex}
\usepackage{multirow}

\newcolumntype{Y}{>{\centering\arraybackslash}X}
\renewcommand{\arraystretch}{2}

\lstset{
	basicstyle=\ttfamily\small,
    keywordstyle=\color{blue}\ttfamily,
    stringstyle=\color{red}\ttfamily,
    commentstyle=\color{green}\ttfamily,
    breaklines=true,
	showstringspaces=false,
	tabsize=2
}


\title{Prototype of Esper-based DAQ expert system}
\date{\today}
\author{Tomasz Bawej}


\begin{document}

\maketitle

\tableofcontents
\clearpage

\section{Overview}
\subsection{Nutshell}
The program examines events, arriving in streams, looking for specified patterns in order to produce an appropriate notification or create new events using transfomed, 
	derived or filtered information.
The events, being arrays of Java Objects, are passed through matching instructions - \emph{statements} - as they arrive and, once processed, they get discarded.
Should the processed events match a given pattern, the information is available right away. 
This is often referred to as an inversed database paradigm with the queries being retained and the data passing through them.

\subsection{Events}
In the project setup events are played back from a database. Upon retrieval of a database row it gets converted into an array of \texttt{Object}s that gets passed to the engine. 
A special \texttt{CurrentTimeSpanEvent} is used to update the engine clock as the events with progressing timestamps are retrieved.

\subsection{Statements}
Statement is an instruction to the engine indicating the way that certain events should be processed. Within the Esper engine each statement is an \texttt{EPStatement} Object that several types of 
listeners can be attached to. Most importantly, a listener can be specified to receive updates from the statement - its output triggered by the most recent event. Such listeners are used to route the
notifications as they are produced by the Esper engine. \\
While the statement objects can be created entirely by the means of invoking particular Esper API methods, Esper provides an SQL-like Event Processing Language (EPL) to facilitate the process.
EPL statements can either be deployed one-by-one by passing a corresponding String literal or as whole files - modules.


\section{Details}
Project comprises the following modules:
\begin{itemize}
	\item LOAD - the application controller object governing the lifecycle of federated objects
	\item EventsProcessor - Esper engine wrapper responsible
	\item FieldsTypeResolver
	\item EventsTap - an interface 
	\begin{itemize}
		\item DbFlashlistEventsTap
	\end{itemize}
	\item EventsSink
\end{itemize}


\section{Use cases}

The project has never moved beyond the prototype phase and there is no deployment mechanism in place or whatsoever.
So far it has only been launched directly from the Eclipse IDE.

The project has never progressed beyond the prototype phase and there is no deployment mechanism in place or whatsoever as so far it has only been launched directly from the Eclipse IDE. Also the term \textit{usecase} refers more to how the code could be reused, rather than to the features of a complete piece of software.

\subsection{Offline analysis}
In order to perform offline analysis, the project has to be configured to use the \emph{Events Database}(see section \ref{subsec:eventsdb} for details) and a \emph{Hardware Configuration Database} connection (see section \ref{subsec:hwconfdb} for details) needs to be set up. An example of corresponding events database configuration is shown below.

\begin{lstlisting}
flashlistDbMode=read
flashlistDbType=mysql
flashlistDbHost=localhost
flashlistDbUser=load
# assumes password-free access for user load, preferably read-only
flashlistDbName=flashlists_rest
\end{lstlisting}


\subsection{Online analysis}
Online analysis requires a slightly different approach: \emph{Events Database} should be disabled in favour of of providing \texttt{onlineFlashlistsRoot} entries pointing to a network location to retrieve the flashlists from (for example see below). 
Connection to the \emph{Hardware Configuration Database}(see section \ref{subsec:hwconfdb} for details) remains a requirement, while a SOCKS proxy (see section \ref{subsec:proxy} for details) configuration is also needed.

\begin{lstlisting}
onlineFlashlistsRoot[0]=http://srv-c2d04-19.cms:9941/urn:xdaq-application:lid=400/
onlineFlashlistsRoot[1]=http://srv-c2d04-19.cms:9942/urn:xdaq-application:lid=400/
\end{lstlisting}


\subsubsection{Dumping online data}
Originally the data was saved to and played back from files only. Thus, no mechanism has been implemented to dump the data directly into a database. Instead, provided that flashlists location is supplied and reachable, their contents are dumped into the folder specified by a corresponding configuration option of the dumper project. The project can be found under the \texttt{extras/flashdumper} subdirectory and is independent from L0AD.
Within the output directory a subdirectory is created for each flashlist type. Flashlist rows are dumped into the files named 0, 1, 2... and so on - unique rows only, switching to the next file once the current one reaches the size of 256MB. A sample output directory structure is depicted below:

\begin{lstlisting}
.
+-- urn:xdaq-flashlist:BU
|   +-- 0
|   +-- 1
|   +-- 2
|   +-- 3
+-- urn:xdaq-flashlist:diskInfo
|   +-- 0
|   +-- 1
+-- urn:xdaq-flashlist:StorageManagerPerformance
    └-- 0

\end{lstlisting}
Dumping also involves adding information about the time of fetching the flashlist - the \texttt{fetchstamp} column.

\subsubsection{Populating database}
Switching \texttt{flashlistDbMode} parameter to \texttt{write} makes the \texttt{Load} class \texttt{main} method invoke the \texttt{main} method of \texttt{MysqlDumper} class that used to be a part of a separate application at some point. A rudimentary \texttt{MongoDumper} class exists as well and during the early tests it was called from \texttt{MysqlDumper} (back then called something else) if MongoDB was the database type specified in the configuration. Work on MongoDB has since been abandoned and that awkward class arrangement has never been refactored. The source has however been retained in case the work with Mongo is resumed some day. \\
Listed below is a sample snippet of the configuration file depicting the database-pumping setup. 

\begin{lstlisting}
flashlistDbMode=write
flashlistDbType=mysql
flashlistDbHost=myDbHost
flashlistDbUser=myDbUser
flashlistDbName=myDbName
flashlistDbPass=myDbPass

flashlistForDbDir[0]=/depot/flashlists13.11/
flashlistForDbDir[1]=/depot/flashlists13.11_2/
flashlistForDbDir[2]=/depot/flashlists13.11_3/
flashlistForDbDir[0]=/depot/flashlists13.11_4/
\end{lstlisting}

The configuration above assumes each of \emph{root} flashlist directories contains a set of subdirectories named after the flashlist they hold the dumps of.

\section{Configuration requirements}
\subsection{RCMS framework} \label{subsec:framework}
The project was confirmed to compile and run with revision 8055 of the \emph{RCMS framework} (\texttt{https://svn.cern.ch/reps/rcms/rcms/trunk/framework}). 
Some later revisions make it impossible to compile the project against the framework.

\subsection{Hardware Configuration Database}
\label{subsec:hwconfdb}
\emph{Hardware Configuration DB connection} is required for the \texttt{HwInfo} class to work. \texttt{HwInfo} provides helper methods that can be used in EPL statements to retrieve information about the hardware.
In development environment the connection was set up using a port-forwarding option in \texttt{~/.ssh/config}: \texttt{LocalForward 10121 cmsrac11-v:10121}, a hard-coded DB url: \texttt{jdbc:oracle:thin:@localhost:10121/cms\_omds\_tunnel.cern.ch} and an active ssh connection with \emph{cmsusr}.

\subsection{SOCKS proxy}
\label{subsec:proxy}
Fetching flashlists as they are published was achieved via configuring ssh (\texttt{DynamicForward 1080} option), specifying the address to fetch the flashlists from and settings for SOCKS Proxy:
\begin{lstlisting}
socksProxyHost=127.0.0.1
proxySet=true
socksProxyPort=1080
\end{lstlisting}
An active ssh connection with \emph{cmsusr} was needed, too.

\subsection{Events Database}
\label{subsec:eventsdb}
The Events Database was set up during development to facilitate events playback, especially any partial playback involving arbitrary start and end timestamp.
Database structure automatically mimicked the structure of published flashlists, i.e. a table was created for each flashlist type and a column for each flashlist field. 
For convenience, all fields were stored as \texttt{VARCHAR} and one additional numeric column was added to each table: \texttt{fetchstamp} indicating the timestamp of fetching the flashlist. An extra table \texttt{fetchstamps} stores all the unique \texttt{fetchstamp} values for all other tables. This minimizes "query misses" - attempts to retrieve data for a \texttt{fetchstamp} with no corresponding events. \\
Also, tables storing useful events (i.e. used for analyses) have been indexed by \texttt{fetchstamp} column. Otherwise the DB performance made the project unusable. 

Password can be specified using the \texttt{flashlistDbPass} parameter. \emph{MySQL} is the only fully supported database type with some rudimentary \emph{MongoDB} also exists, but needs to be worked on to be useful.








\subsection{Overview}
\subsection{Details}

Until the proper, Esper-powered approach was taken, resolving EPL dependencies used to be handled by the means of inserting extra \texttt{#include} statements and determining the deployment order by hand.
Some remnants of that are present in the EPL loading code or in the EPL itself and should have been removed.

\subsection{Implemented checks}
A summary of implemented checks along with some brief notes can be found in an xml file under the \texttt{epl/xml}

%\section{Overview of implemented checks}


\begin{tabularx}{1\textwidth}{|*{2}{Y|}}
\hline
Purpose of the check    	& Solution notes / related EPL files \\
\hline
\multicolumn{2}{|c|}{During an ongoing run.} \\
%\cline{1-3}
\hline
	
	\multirow{2}{0.5000\textwidth}{Message on Run Start giving: Run NR, SID, Detectors in, Feds in per detector} & 
	Subsystems in uses a custom aggretaion method that turns out to be running all the time (and not on demand) \\
	\cline{2-2}
	& \small{\texttt{runStartStop.epl}} \\

	\hline
	\multirow{2}{0.5000\textwidth}{Message on Run Stop giving: Run Nr, SID, Detectors in, (*)Feds in per detector, avg L1 rate, avg stream A rate, avg dead time (from trigger LAS), duration} &
	Same as for run start, but also: get avg L1 rate, avg stream A rate, avg dead time from trigger LAS, the duration \\
	\cline{2-2}
 	& \small{\texttt{runStartStop.epl}} \\

	\hline
	\multirow{2}{0.5000\textwidth}{Message on L1 trigger rate jump of 10\% or more} &
	EPL file creates some windows and streams useful for performing other loigic. A javascript method needs to be raplaced with a case statement \\
	\cline{2-2}
	& \small{\texttt{level1TriggerRate.epl}} \\

	\hline
	\multirow{2}{0.5000\textwidth}{Message on subsys going to any of: ERROR, RUNNING\_DEGRADED, RUNNING\_SOFT\_ERROR\_DETECTED, PAUSING, PAUSED, RESUMING} & \\
	\cline{2-2}
	& \small{\texttt{subsystemsStateChanges.epl}} \\

	\hline
	\multirow{2}{0.5000\textwidth}{Message on any state change of DAQ} & \\
	\cline{2-2}
	& subsystemsStateChanges.epl \\
	
	\hline
	\multirow{2}{0.5000\textwidth}{Message on jobcontrol flashlist not being updated after 1 minute} & \\
	\cline{2-2}
	& jobcontrolNotUpdated.epl \\
	
	\hline
	\multirow{2}{0.5000\textwidth}{FED dead-time $>$ 1\%} & Investigates backpressure on the same FED or its main FED. \\
	\cline{2-2}
	& deadtimeAndBackpressure.epl \\

	\hline
	\multirow{2}{0.5000\textwidth}{Backpressure $>$ 1\%} & \\
	\cline{2-2}
	& deadtimeAndBackpressure.epl \\

	\hline
	\multirow{2}{0.5000\textwidth}{Message on stream A $>$ 500 Hz} &
	Simplified, does not satisfy: after 10 seconds, repeat message every 10 seconds. Need to sum the last-per-context values \\
	\cline{2-2}
	& streamARate.epl \\

	\hline
	\multirow{2}{0.5000\textwidth}{FEDs fraction other} &
	A FedFractions stream is constantly filled with derivatives computed for each subsequently arriving pair of FmmInput events with the same fedId. 
	Fractions (busy+warning	+error+ready+oos) must add up to one, otherwise FED spends some time in an illegal (other) state. This fact should be reported and using integrals is preffered for calculations to avoid floating point numbers comparison. Why does it report negative fractions at the beginning? (As well as negative dTime). Never tested on positives. \\
	\cline{2-2}
	& fedFractions.epl \\
\hline
	\multicolumn{2}{|c|}{Rate 0 for $>$ 10 seconds} \\
	
	
	\hline
	\multirow{2}{0.5000\textwidth}{Check Bx alignment and print message if not aligned repeat after 10 seconds} & Fails to repeat the message every 10 seconds if the problem persists\\
	\cline{2-2}
	&  rateZeroTests.epl\\

	\hline
	\multirow{2}{0.5000\textwidth}{Check triggers(events) alignment + print message if not aligned repeat after 10 seconds} & 
		Fails to repeat the message every 10 seconds if the problem persists \\
	\cline{2-2}
	& rateZeroTests.epl \\

	\hline	
	\multirow{2}{0.5000\textwidth}{FEDs stuck in ERROR/OOS/WARNING/BUSY(anything not READY)} & 
	Written, running, never seen working yet. Using the FedFractions stream to find events of interest and SuspendedStatements window to suspend/resume the statement upon reception of RateStuckAtZeroEvent or RateFineEvent defined in the same source file as a way to broadcast the message that the expert enters a state where the run is ongoing but the rate has been 0 for long enough or that the rate is fine and experiment is running. 
ERROR/OOS/WARNING/BUSY have their corresponding fractions on FMMInput. Stuck means fraction = 1. A FED might also be stuck in an other state. Use the integrals from two consecutive events to determine that.
\\
	\cline{2-2}
	& rateZeroTests.epl\\


	\hline
	\multirow{2}{0.5000\textwidth}{List FEDs with backpressure or deadtime} & \\
	\cline{2-2}
	& rateZeroTests.epl\\


	\hline
	\multirow{2}{0.5000\textwidth}{Check if number of resyncs and the last resync event number is the same in all FEDs?} & 
	Check if the number of resyncs is the same and check if the last event (the resync was seen for) is the same. myrinetResync - number of resync events, myrinetLastResyncEvt - last event that the resync was seen for.\\
	\cline{2-2}
	& rateZeroTests.epl\\


\end{tabularx}



A 'mock' version (with additional comments) of the table below bac be found under the \texttt{epl/xml} directory.









\end{document}
