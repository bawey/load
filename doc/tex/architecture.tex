\subsection{Main components}

Program's entry point is the \texttt{main} method of the class \texttt{Load}. Except for that, the project consists of several core classes and further helpers.
The core classes include:
\begin{itemize}
	\item \texttt{EventProcessor} - a wrapper for the Esper event processing engine. Facilitates some manipulations and performs the setup (including registration of helper methods available from the EPL level).
	\item \texttt{EventsTap} - an abstract class extended only by \texttt{DataBaseFlashlistEventsTap} and \texttt{OnlineFlashlistEventsTap}. Depending on the database read/write mode, a different one is registered during setup.
	\item \texttt{EventsSink} - named for the sake of symmetry with the tap, provides an abstract base class for producing the output. Multiple sinks can be registered in the same time to write into files, display something on the screen etc. Comes with, mmost notably, \texttt{SwingGui} and \texttt{FileSink} subclasses.
	\item \texttt{EplProvider} - only has two implementations: \texttt{BasicStructEplProvider} and \texttt{FileBasedEplProvider} and only the latter is actually useful as of the time of writing this summary. \texttt{EplProvider}s do just what the name suggests: register statements and thus instruct the engine what to do with the events.

\end{itemize}

The helpers include, among the others:
\begin{itemize}
	\item \texttt{FieldTypeResolver}
	\item \texttt{Trx} - a set of convenience methods callable from EPL statements
	\item \texttt{HwInfo} - as above, yet related to CMS hardware
	\item \texttt{Settings} - a global singleton for handling the configuration values
	\item annotations \texttt{@Verbose} and \texttt{@Watched} play a role of the missing link between the engine and \texttt{EventSink}s. They can be added to a statement (along with some extra parameters), so that the statements later on automatically get some listeners assigned. These in turn forward the output to \texttt{EventSink}s
\end{itemize}



\subsection{Configuration properties}

\texttt{load.properties} file explained:

\subsubsection{Proxy parameters}
\texttt{socksProxyHost=127.0.0.1}
\texttt{proxySet=true}
\texttt{socksProxyPort=1080}
\subsubsection{Flashlist database settings}
\texttt{flashlistDbMode=read}
\texttt{flashlistDbType=mysql}
\texttt{flashlistDbHost=localhost}
\texttt{flashlistDbUser=load}
\texttt{flashlistDbName=flashlists\_rest}
\texttt{flashlistDbEngine=myisam}
\texttt{retrievalTimestampName=fetchstamp}
\texttt{flashlistDbIndexTimestamps=true}

\subsubsection{Online flashllists}

\texttt{onlineFlashlistsRoot[0]=http://srv-c2d04-19.cms:9941/urn:xdaq-application:lid=400/}
\texttt{onlineFlashlistsRoot[1]=http://srv-c2d04-19.cms:9942/urn:xdaq-application:lid=400/}

\subsubsection{Dumped flashlists}
\texttt{flashlistForDbDir[0]=/depot/flashlists13.11/}
\texttt{flashlistForDbDir[1]=/depot/flashlists13.11\_2/}
\texttt{flashlistForDbDir[2]=/depot/flashlists13.11\_3/}
\texttt{flashlistForDbDir[0]=/depot/flashlists13.11\_4/}
\texttt{flashlistForDbDir[0]=/home/bawey/Desktop/flmini/flashlistsExport}
\texttt{flashlistForDbDir[2]=/home/bawey/Desktop/flashlists/41}


\subsubsection{EPL directories}
\texttt{eplDir[0]=epl}


\subsubsection{Output modules}
\texttt{view[0]=ThroughputMonitor}
\texttt{view[1]=FileSink}
\texttt{view[2]=SwingGui}


\subsubsection{Events playback range}

\texttt{timerStart=1383763860000}
\texttt{timerEnd=1383763860800}

dateFormat=yyyy-MM-dd'T'HH:mm:ss.SSS

outputDir=/home/bawey/load/

flashlists=levelZeroFM\_subsys;EVM;frlcontrollerLink;frlcontrollerCard;levelZeroFM\_static;gt\_cell\_lumiseg;FMMInput;jobcontrol;EventProcessorStatus;StorageManagerPerformance;FMMStatus;hostInfo

blacklist\_jobcontrol=jobTable
blacklist\_frlcontrollerCard=myrinetProcFile,myrinetBadEventNumber,myrinetBadHeaderMark,myrinetBadSegmentNumber,myrinetBadTrailerMark,myrinetBlock,myrinetFEDBadCRC,pendingTriggers,freeBlockCount,instance
blacklist\_levelZeroFM\_subsys=FEDS
blacklist\_StorageManagerPerformance=activeEPs;averagingTime;bandwidthPerStream;bandwidthToDisk;closedFiles;connectedEPs;connectedRBs;copyWorkers;dataEvents;discardedDQMEvents;diskPaths;diskWriterBusy;dqmConsumers;dqmEventProcessorBusy;dqmFoldersPerEP;dqmQueueBandwidth;dqmQueueRate;entriesInDQMQueue;entriesInFragmentQueue;entriesInStreamQueue;errorEvents;eventConsumers;eventsPerStream;faultyEvents;fragmentProcessorBusy;fragmentQueueBandwidth;fragmentQueueRate;fragmentStoreMemoryUsed;fragmentStoreSize;ignoredDiscards;injectWorkers;instantBandwidth;instantRate;lid;memoryUsedInDQMQueue;memoryUsedInFragmentQueue;memoryUsedInStreamQueue;numberOfDisks;openFiles;outstandingDataDiscards;outstandingDQMDiscards;poolUsage;processedDQMEvents;receivedFrames;runNumber;sataBeastStatus;stateName;storedEvents;storedVolume;streamQueueBandwidth;streamQueueRate;totalDiskSpace;unwantedEvents;usedDiskSpace;writtenEventsBandwidth;writtenEventsRate

logSinks="ch.cern.cms.load.sinks.SwingGui";


Connecting to databas

Dumping data
The very main method of the program allows switching into dumping mode:

\begin{lstlisting}
	public static final void main(String[] args) {
   	 Thread.currentThread().setName("Level 0 Anomaly Detective");
   	 instance = getInstance();
   	 if (instance.settings.getProperty(DataBaseFlashlistEventsTap.KEY\_DB\_MODE, "read").equalsIgnoreCase("write")) {
   		 MysqlDumper.main(args);
   	 } else {
   		 instance.defaultSetup();
   	 }
    }
\end{lstlisting}

Which in fact was an ugly way to merge the original dumper into the project. Anyway, it might still work.

HwInfo dumping
During development it was possible to dump the HwInfo database into file and load it afterwards to speed the program launch up. It however required several modifications to the framework (making all the classes involved in dumping implement Serializable) that have not been introduced into it. Hence the HwInfo class still examines the settings for options specifying the HwInfo dump file to write to / read from, yet without providing these it will simply use the remote DB.
EPL annotations
As mentioned above, L0AD introduces two annotations that can be used in EPL statements:
@Verbose
@Watched
The options they take is described in the source code.

EPL statements
(Please note: for the statements to be effective, the events have to be defined)

Event definitions
Event definitions are dynamically fed into the program every time it starts up, based on the 

Data types resolving
The FieldTypeResolver class stores information about the types that particular field should be converted to. This can be specified both globally, for all fields of given name, or locally for particular event type. Unfortunately, these rules are currently hard-coded into the class.
Implemented checks



